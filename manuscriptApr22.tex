%% LyX 2.2.4 created this file.  For more info, see http://www.lyx.org/.
%% Do not edit unless you really know what you are doing.
\documentclass[12pt,times]{extarticle}
\usepackage{mathptmx}
\usepackage[T1]{fontenc}
\usepackage[latin9]{inputenc}
\usepackage{geometry}
\geometry{verbose,tmargin=2.5cm,bmargin=2.5cm,lmargin=2.5cm,rmargin=2.5cm}
\usepackage{color}
\usepackage{units}
\usepackage{url}
\usepackage{amsmath}
\usepackage{amssymb}
\usepackage{graphicx}
\usepackage[authoryear]{natbib}

\makeatletter

%%%%%%%%%%%%%%%%%%%%%%%%%%%%%% LyX specific LaTeX commands.
%% Because html converters don't know tabularnewline
\providecommand{\tabularnewline}{\\}
%% A simple dot to overcome graphicx limitations
\newcommand{\lyxdot}{.}


%%%%%%%%%%%%%%%%%%%%%%%%%%%%%% User specified LaTeX commands.
%\newcommand{\nicefrac}[2]{\ensuremath ^{#1}\!\!/\!_{#2}}
\usepackage{fancybox}
\usepackage[export]{adjustbox}

\usepackage{todonotes}
%\usepackage{afterpage}

%\usepackage[switch]{lineno}

\usepackage[colorlinks,bookmarksopen,bookmarksnumbered,citecolor=red,urlcolor=red]{hyperref}

\newcommand\BibTeX{{\rmfamily B\kern-.05em \textsc{i\kern-.025em b}\kern-.08em
T\kern-.1667em\lower.7ex\hbox{E}\kern-.125emX}}

%\usepackage{moreverb}

\usepackage{flushend}

\makeatother

\begin{document}

\title{Adaptively Implicit MPDATA Advection for Arbitrary Courant numbers
and Meshes}

\author{Hilary Weller, James Woodfield, Christian K\"uhnlein and Piotr K.
Smolarkiewicz}
\maketitle

\section{Introduction}

\subsection*{Motivation}

Time step restrictions based on advection have always posed a problem
for models of the atmosphere. The Courant-Friedrichs-Lewy (CFL) condition
states that explicit Eulerian schemes will have time step restrictions
based on the size of the domain of dependence. Typically this means
that explicit schemes cannot run with a Courant number greater than
one or thereabouts. This is equivalent to saying that an advected
quantity cannot be moved by more than one mesh cell (or grid box)
in one time step.

CFL constraints can be particularly cruel where resolution is higher
than it needs to be for accuracy, such as near the poles of a latitude-longitude
mesh. An early work around was to use polar filtering; artificially
removing oscillations near the poles where mesh lines converge \citep{CD91}.
Polar filtering has been improved and used more recently in magneto-hydro-dynamics
by \citet{ZSL+19}. However polar filtering led to parallel scaling
bottlenecks and mistrust of solutions near the poles in models of
the atmosphere. The UK Met Office gained accuracy and efficiency by
replacing their model employing polar filtering with a semi-implicit
, semi-Lagrangian (SISL) model to avoid time step restrictions on
a latitude-longitude mesh \citep{DCM+05}. SISL eases time step constraints
by treating acoustic and gravity waves implicitly while the advection
is solved with the semi-Lagrangian method which is stable, accurate
and efficient for long time steps but not conservative. The lack of
conservation is regarded as inadequate for climate modelling and is
associated with spurious features such as eternal fountains which
involve a positive feedback loop that creates moisture in convectively
unstable columns \citet{ZA20}.

The requirement to run stably with large Courant numbers is less severe
now that weather and climate models have largely moved away from latitude-longitude
meshes in favour of quasi-uniform meshes such as the cubed sphere
and icosahedral meshes \citep{UJK+17}. However, the problem remains
severe in the vertical direction where mesh spacing can be fine and
large vertical velocities can occur when atmospheric convection is
resolved. A solution in the vertical direction is to use Lagrangian
floating levels \citep{Lin04} that are conservatively mapped back
to the fixed Eulerian mesh. Lagrangian floating levels are equivalent
to conservative semi-Lagrangian \citep{HLM11} which can be made efficient
for large Courant numbers in one dimension and consequently can work
on tensor product meshes \citep[eg.][]{LLM96}. These are related
to the Arbitrary Lagrangian-Eulerian (ALE) method \citep[eg. ][]{HAC97}
which solves equations in a Lagrangian frame and them remaps the solution
back to the original mesh. However ALE methods suffer from time step
restrictions associated with avoiding mesh distortions.

Due to the difficulties in allowing long time steps while maintaining
exact local conservation on arbitrary meshes, it is worth considering
implicit time stepping for advection. The aim of this paper is to
present an advection scheme with the following properties:
\begin{enumerate}
\item Stable for very large Courant numbers.
\item Applicable on arbitrary meshes.
\item Similar cost and accuracy to explicit schemes when the Courant number
is below one. 
\item Monotonic when required.
\item At least first-order accurate where the Courant number is large.
\item Multi-tracer efficient.
\item Good parallel scaling.
\end{enumerate}
This paper proposes the use of blended implicit/explicit time stepping
combined with the MPDATA advection scheme (multidimensional positive
definite advection transport algorithm). However the methods for combining
implicit and explicit and the method for achieving monotonicity should
translate to other explicit advection schemes.

\subsection*{Background on MPDATA}

MPDATA was introduced by \citet{Smol83,Smol84} and reviewed by \citet{SM98}.
MPDATA consists of a first-order upwind, explicit step to solve the
advection equation followed by a correction step to remove the first-order
error. An anti-diffusive velocity is calculated and the solution of
the first-order step is corrected by advecting it with the anti-diffusive
velocity. The corrective transport is also applied using explicit
first-order upwind differencing so the solution remains positive definite.
The anti-diffusive velocity is necessarily divergent so the scheme
is not bounded above. The resulting scheme is second-order accurate
in space and time and stable for a Courant number less than one. \citet{SS05b}
extended MPDATA to unstructured meshes while retaining second-order
accuracy.

In order to use MPDATA to advect vectors and other signed quantities
and also to improve accuracy, MPDATA gauge effectively advects a quantity
plus a constant in order to move values away from zero \citep{SC86},
meaning that the advected quantity is no longer bounded below by zero.
Monotonicity can be enforced using flux-corrected transport \citep{SG90,Zal79}.
\citet{KS17} combine the temporal correction with the divergence
correction to create an anti-diffusive velocity that is defined only
in terms of fluxes.

\subsection*{Background on Implicit Time Stepping}

Implicit time stepping is ubiquitous in atmospheric modelling for
solving the terms of the equations of motion responsible for fast
waves such as gravity and acoustic waves. However, implicit time stepping
has been used very little for advection in atmosphere and ocean modelling.
Implicit time stepping for advection in the mathematics and engineering
literature will be discussed first and then we will return to its
uses to date in atmosphere and ocean modelling.

Implicit time stepping for advection has a severe order barrier; no
implicit method exists with order greater than one which is monotonic
for all time steps \citep{GST01}. Higher order implicit multi-stage
(Runge-Kutta) and multi-step schemes exist that are unconditionally
linearly stable but if we additionally require monotonicity (no new
extrema generated), then higher-order implicit methods have time step
restrictions, known as radii of monotonicity. As with spatial discretisation,
high order implicit time stepping has been non-linearly combined with
first order implicit time stepping to try to break the order barrier
and increase the radius of monotoncity. \citet{YWH85,Yee87} proposed
implicit TVD (total variation diminishing) schemes for solving hyperbolic
equations and achieved high accuracy for large Courant numbers. However
they could not find a second-order scheme that was guaranteed both
TVD and conservative for non-constant coefficient advection. \citet{MB17}
used flux corrected transport \citep[FCT,][]{Zal79} to improve temporal
accuracy in implicitly solved small cells without generating new extrema. 

In atmospheric modelling, implicit time stepping for advection has
been used to treat small, cut-cells stably at modest time steps \citep[e.g.][]{JKW11}
and more recently to treat isolated strong updrafts stably \citep{WS20}.
\citet{WS20} and \citet{Shch15} avoided the order barriers by limiting
order of accuracy to first where ever implicit time stepping for advection
was used. The lack of accuracy was not considered problematic because
of the sparsity of the use of implicit advection. \citet{CWPS17}
compared implicit advection with dimensionally split, flux-form semi-Lagrangian
advection and found that the dimensionally split scheme was more accurate
and more efficient than implicit advection for all Courant numbers.
However, this was not a like for like comparison; the dimensionally
split scheme was a higher-order accurate scheme, limited to tensor
product meshes and suffered from mesh imprinting errors on distorted
meshes. 

\section{An Implicit/Explicit MPDATA}

The version of explicit MPDATA and the blended implicit/explicit MPDATA
defined here are implemented using the OpenFOAM library (\url{https://openfoam.org/})
using standard OpenFOAM operators and linear equation solvers. The
code is available as part of the AtmosFOAM repository (\url{https://github.com/AtmosFOAM/})
compiled with OpenFOAM7. 

\subsection{Explicit MPDATA on an Arbitrary Mesh}

\begin{figure}
\noindent \begin{centering}
\scalebox{0.5}[0.5]{\input{figures/cells.pdftex_t}}
\par\end{centering}
\caption{Two cells in an arbitrary mesh. $\mathbf{x}_{C}$ and $\mathbf{x}_{N}$
are the cell centres of cell $C$ and its neighbour, $N$, over face
$f$ and $\mathbf{x}_{f}$ is the face centre. $\mathbf{S}_{f}$ is
the face area vector, normal to face $f$ with magnitude equal to
the face area. $\mathbf{u}$ is the velocity and $\mathbf{x}_{d}$
is the departure point for face $f$ at time $n+\nicefrac{1}{2}$
(i.e. the centre of the volume swept through the face between times
$n$ and $n+1$).\label{fig:cells}}
\end{figure}

The description of the explicit scheme is consistent with basic MPDATA
principles \citep[e.g.][]{SS05b}, but introduces a novel derivation
based on a flux-form semi-Lagrangian method and assuming an arbitrary
mesh in Cartesian co-ordinates rather than using co-ordinate transforms.
We will describe MPDATA for solving the linear advection equation
for advected quantity $\psi$ with velocity field $\mathbf{u}$:
\begin{equation}
\frac{\partial\psi}{\partial t}+\nabla\cdot\left(\mathbf{u}\psi\right)=0.\label{eq:linearAdvection}
\end{equation}
This is solved using Gauss's divergence theorem on an arbitrary mesh
to go from time level $n$ to $n+1$ a time step $\Delta t$ apart:
\begin{equation}
\psi_{C}^{n+1}=\psi_{C}^{n}-\frac{\Delta t}{V_{C}}\sum_{f\in C}\psi_{f}^{n+\nicefrac{1}{2}}U_{f}\label{eq:advectionGauss}
\end{equation}
where $\psi_{C}$ is the cell mean value of $\psi$ in cell $C$,
$V_{C}$ is the volume of cell $C$, $f\in C$ are the faces of cell
$C$, $\psi_{f}^{n+\nicefrac{1}{2}}$ is the value of $\psi$ at face
$f$ at time $n+\nicefrac{1}{2}$, $\mathbf{u}_{f}$ the velocity
at face $f$ and $\mathbf{S}_{f}$ is the face area vector \textemdash{}
the outward pointing vector normal to face $f$ with magnitude equal
to the area of face $f$ (fig \ref{fig:cells}). $U_{f}=\mathbf{u}_{f}\cdot\mathbf{S}_{f}$
is the volume flux over face $f$. In this derivation of MPDATA, $\psi_{f}$
is evaluated at the departure point of the face centre at time $n$.
The departure point, $\mathbf{x}_{d}$, is the centre of the volume
swept through the face between times $n$ and $n+1$ which is approximated
by the point a distance $\mathbf{u}\Delta t/2$ upstream of the face
centre:
\begin{equation}
\mathbf{x}_{d}=\mathbf{x}_{f}-\frac{\Delta t}{2}\mathbf{u}_{f}+O\left(\Delta t\right)^{2}\label{eq:departure}
\end{equation}
where $\mathbf{x}_{f}$ is the face centre (see fig \ref{fig:cells}).
The velocity, $\mathbf{u}_{f}$, is evaluated at time $n+\nicefrac{1}{2}$
at the face. In this paper the wind is prescribed. In a more complete
model, $\mathbf{u}_{f}^{n+\nicefrac{1}{2}}$ would be evaluated from
velocities at known positions and times. The dependent variable, $\psi$,
is evaluated at the departure point using the upwind cell centre value
of $\psi$, the gradient of $\psi$ and the velocity divergence:
\begin{equation}
\psi_{f}^{n+\nicefrac{1}{2}}=\psi_{d}^{n}=\psi_{\text{up}}^{n}+\left(\mathbf{x}_{d}-\mathbf{x}_{\text{up}}\right)\cdot\nabla\psi^{n}-\frac{\Delta t}{2}\psi_{\text{up}}^{n}\nabla\cdot\mathbf{u}+O\left(\Delta x^{2},\Delta t^{2}\right)\label{eq:PsiDep}
\end{equation}
where $\psi_{\text{up}}$ is the value of $\psi$ in the cell upwind
of face $f$. $\nabla\psi$ can be calculated in cells ($C$) to second-order
accuracy on an arbitrary mesh using a least-squares approximation:
\begin{equation}
\nabla\psi=\sum_{f\in C}\mathbf{g}_{f}\left(\psi_{N}-\psi_{C}\right)
\end{equation}
where $\psi_{N}$ is $\psi$ in the neighbour of cell $C$ across
face $f$ and where $\mathbf{g}_{f}$ is a vector calculated for each
face of cell $C$ based entirely on the local mesh geometry:
\begin{eqnarray*}
\mathbf{g}_{f} & = & \left(1-w_{f}\right)\frac{|\mathbf{S}_{f}|}{|\mathbf{x}_{N}-\mathbf{x}_{C}|^{2}}D_{C}^{-1}\left(\mathbf{x}_{N}-\mathbf{x}_{C}\right)\\
\text{where }w_{f} & = & \frac{|\mathbf{S}_{f}\cdot(\mathbf{x}_{N}-\mathbf{x}_{f})|}{|\mathbf{S}_{f}\cdot(\mathbf{x}_{N}-\mathbf{x}_{f})|+|\mathbf{S}_{f}\cdot(\mathbf{x}_{f}-\mathbf{x}_{C})|}\ \text{(interpolation weights)}\\
\text{where }D_{C} & = & \sum_{f\in C}\left(1-w_{f}\right)\frac{|\mathbf{S}_{f}|}{|\mathbf{x}_{N}-\mathbf{x}_{C}|^{2}}\left(\mathbf{x}_{N}-\mathbf{x}_{C}\right)\left(\mathbf{x}_{N}-\mathbf{x}_{C}\right)^{T}
\end{eqnarray*}
This is the least squares gradient implemented in the OpenFOAM library.
Equations (\ref{eq:departure}) and (\ref{eq:PsiDep}) are substituted
in to (\ref{eq:advectionGauss}) to give a scheme that is second-order
accurate in space and time but not positive definite, equivalent to
a Law-Wendroff scheme:
\begin{eqnarray}
\psi_{C}^{n+1} & \approx & \psi_{C}^{n}-\underbrace{\frac{\Delta t}{V_{C}}\sum_{f\in C}\psi_{\text{up}}^{n}U_{f}}_{\text{explicit upwind}}-\underbrace{\frac{\Delta t}{V_{C}}\sum_{f\in C}\left(\mathbf{x}_{f}\!-\!\mathbf{x}_{\text{up}}\right)\cdot\left(\nabla\psi^{n}\right)U_{f}}_{\text{spatial corection}}+\underbrace{\frac{\Delta t^{2}}{2V_{C}}\sum_{f\in C}\left(\nabla\cdot\left(\mathbf{u}\psi^{n}\right)\right)U_{f}}_{\text{temporal correction}}\label{eq:advectionGauss-1}
\end{eqnarray}
The explicit upwind term does not introduce unbounded solutions. In
order for the second-order and divergence correction terms to maintain
positivity, they are written as explicit upwind advection using an
anti-diffusive flux, $V_{f}=\mathbf{v}_{f}\cdot\mathbf{S}_{f}$, which
transforms the scheme from Lax-Wendroff to MPDATA and is written in
two stages:
\begin{eqnarray}
\text{explicit diffusive step: }\psi_{C}^{d} & = & \psi_{C}^{n}-\frac{\Delta t}{V_{C}}\sum_{f\in C}\psi_{\text{up}}^{n}U_{f}\\
\text{explicit correction: }\psi_{C}^{n+1} & = & \psi_{C}^{d}-\frac{\Delta t}{V_{C}}\sum_{f\in C}\psi_{\text{vup}}^{d}V_{f}\\
\text{where }V_{f}=\mathbf{v}_{f}\cdot\mathbf{S}_{f} & = & \frac{U_{f}}{\psi_{\text{denom}}}\left\{ \left(\mathbf{x}_{f}-\mathbf{x}_{\text{up}}\right)\cdot\nabla\psi-\frac{\Delta t}{2}\nabla\cdot\left(\psi\mathbf{u}\right)\right\} \label{eq:antiDfluxExp}
\end{eqnarray}
and where $\psi_{\text{vup}}$ is $\psi$ in the upwind cell where
the upwind direction is defined by the sign of $V_{f}$. As $\psi$
is always positive, there is no ambiguity in the sign of $V_{f}$.
The anti-diffusive flux, $V_{f}$, can be calculated iteratively,
using first $\psi^{d}$ and subsequent iterations use the most up
to date version of $\psi^{n+1}$. Eqn (\ref{eq:antiDfluxExp}) is
a continuous version of the expression for the anti-diffusive velocity
in eqn (13) of \citet{SS05b}. The discretisaions of gradients, divergences
and $\psi_{\text{denom}}$ described here give exactly the same expression
as \citet{SS05b} on a uniform structured grid but the treatment of
non-uniformity and skewness are slightly different here.

The explicit diffusive step has a Courant number limit of half for
a divergent velocity field and one for a non-divergent velocity field.
This can be seen by writing the advection as the sum of outward and
inward fluxes:
\begin{eqnarray}
\psi_{C}^{d} & = & \psi_{C}^{n}-\frac{\Delta t}{V_{C}}\sum_{U_{f}>0}\psi_{C}^{n}U_{f}+\frac{\Delta t}{V_{C}}\sum_{U_{f}<0}\psi_{N}^{n}|U_{f}|\\
 & = & \psi_{C}^{n}\left\{ 1-\frac{\Delta t}{V_{C}}\sum_{U_{f}>0}U_{f}\right\} +\frac{\Delta t}{V_{C}}\sum_{U_{f}<0}\psi_{N}^{n}|U_{f}|.
\end{eqnarray}
For a divergent velocity field, $\psi_{C}^{d}$ is positive when the
term in curly brackets is non-negative which happens when $\frac{\Delta t}{V_{C}}\sum_{U_{f}>0}U_{f}\le1$.
The Courant number on an arbitrary mesh is defined as
\begin{equation}
c=\frac{1}{2}\frac{\Delta t}{V_{C}}\sum_{f}|U_{f}|\label{eq:Courant}
\end{equation}
so a Courant number limit of half guarantees positivity. For a non-divergent
velocity field:
\begin{equation}
\sum_{U_{f}>0}U_{f}=\sum_{U_{f}<0}|U_{f}|=c
\end{equation}
so the solution is bounded when the Courant number is less than one.

For the explicit correction to be positive definite and stable, the
anti-diffusive velocity needs to obey the same Courant number restriction.
Considering the case with uniform velocity in one dimension and uniform
$\nabla\psi$ and remembering that $\psi\ge0$ and $c\in[0,1]$, this
is equivalent to
\begin{eqnarray*}
c\left(1-c\right)\frac{|\psi_{C}-\psi_{N}|}{2\psi_{\text{denom}}} & \le & 1.
\end{eqnarray*}
Following \citet{SS05b}
\begin{equation}
\psi_{\text{denom}}=\frac{1}{2}\left(\psi_{C}+\psi_{N}+\varepsilon\right)
\end{equation}
where $\varepsilon$ is a small number. This ensures that explicit
MPDATA is positive definite. The results presented here use $\varepsilon=10^{-16}$.

\subsection{Blended Implicit/Explicit MPDATA\label{subsec:MPDATA_theta}}

We describe a blended implicit/explicit scheme to be a generalisation
of Crank-Nicolson (or trapezoidal implicit) with off-centering $\theta$
which can vary in space and is defined on faces (for conservation):
\begin{equation}
\psi_{C}^{n+1}=\psi_{C}^{n}-\frac{\Delta t}{V_{C}}\sum_{f\in C}\left\{ \left(1-\theta_{f}\right)\psi_{f}^{n}+\theta_{f}\psi_{f}^{n+1}\right\} U_{f}.\label{eq:advectionGauss-2}
\end{equation}
This is second order in time only for $\theta=\frac{1}{2}$. We will
find the MPDATA anti-diffusive flux that corrects a scheme which is
first-order accurate in space and off-centered by $\theta$ in time.
So the first step, before the MPDATA correction, is:
\begin{equation}
\psi_{C}^{d}=\psi_{C}^{n}-\frac{\Delta t}{V_{C}}\sum_{f\in C}\left\{ \left(1-\theta_{f}\right)\psi_{\text{up}}^{n}+\theta_{f}\psi_{\text{up}}^{d}\right\} U_{f}.\label{eq:thetaUp}
\end{equation}
The proof that this first step gives positive solutions and bounded
solutions for non-divergent velocity fields on arbitrary meshes is
in appendix \ref{sec:appxBounded}.

\begin{figure}
\noindent \begin{centering}
\scalebox{0.5}[0.5]{\input{figures/departureArrival.pdftex_t}} \scalebox{0.4}[0.4]{\input{figures/departureArrivalLargec.pdftex_t}}
\par\end{centering}
\caption{The volume that is swept through face $f$ in one time step and the
departure and arrival points, $\mathbf{x}_{d}$ and $\mathbf{x}_{a}$
for small (left) and large (right) Courant numbers.\label{fig:departureArrival}}
\end{figure}

To find the second-order approximation of $\psi_{f}^{n+1/2}$, we
consider a linear combination of $\psi$ at the departure point at
time $n$ and $\psi$ at the arrival point at time $n+1$:
\begin{equation}
\psi_{f}^{n+1/2}=\left(1-\theta_{f}\right)\psi_{d}^{n}+\theta_{f}\psi_{a}^{n+1}\label{eq:thetaPsiFace}
\end{equation}
where the locations of the departure and arrival points are shown
in fig \ref{fig:departureArrival} and are given by
\begin{eqnarray*}
\mathbf{x}_{d} & = & \mathbf{x}_{f}-\frac{\Delta t}{2}\mathbf{u}_{f}\\
\mathbf{x}_{a} & = & \mathbf{x}_{f}+\frac{\Delta t}{2}\mathbf{u}_{f}
\end{eqnarray*}
The values of $\psi$ and the departure and arrival points can be
approximated by
\begin{eqnarray*}
\psi_{d}^{n} & = & \psi_{\text{up}}^{n}+\left(\mathbf{x}_{d}-\mathbf{x}_{u}\right)\cdot\nabla\psi\\
\psi_{a}^{n+1} & = & \psi_{\text{up}}^{n+1}+\left(\mathbf{x}_{a}-\mathbf{x}_{u}\right)\cdot\nabla\psi.
\end{eqnarray*}
Substituting these into (\ref{eq:thetaPsiFace}) gives
\begin{eqnarray}
\psi_{f}^{n+\nicefrac{1}{2}} & = & \left(1-\theta_{f}\right)\psi_{\text{up}}^{n}+\theta_{f}\psi_{\text{up}}^{n+1}+\left(\mathbf{x}_{d}-\mathbf{x}_{\text{up}}\right)\cdot\nabla\psi^{n}-\left(1-2\theta_{f}\right)\frac{\Delta t}{2}\mathbf{u}\cdot\nabla\psi\label{eq:PsiDep-1}
\end{eqnarray}
where $\mathbf{x}_{u}$ is the centre of the cell upwind of face $f$.
This correction is not stable for Courant number, $c>2$ or $\theta>\frac{1}{2}$
(appendix \eqref{sec:appxStability}). For stability for all $c$
and second-order accuracy where $\theta\le\frac{1}{2}$ the correction
step is:
\begin{eqnarray}
\psi_{C}^{n+1} & = & \psi_{C}^{d}+\frac{\Delta t}{V_{C}}\sum_{f\in C}\psi_{\text{vup}}^{d}V_{f}\\
\text{where }V_{f} & = & \mathbf{v}_{f}\cdot\mathbf{S}_{f}=\frac{U_{f}}{\psi_{\text{denom}}}\left\{ \left(\mathbf{x}_{f}-\mathbf{x}_{u}\right)\cdot\nabla\psi-\chi\frac{\Delta t}{2}\mathbf{u}\cdot\nabla\psi\right\} \label{eq:Vcorr}\\
\chi & = & \max\left(1-2\theta_{f},\ 0\right)
\end{eqnarray}
This gives a first-order error in time for $\theta>\frac{1}{2}$ which
is only used for large Courant numbers ($>2$). It is stable on a
uniform one-dimensional grid (appendix \eqref{sec:appxStability})
but on an arbitrary grid, some smoothing is needed when $\theta>0$
(section \ref{subsec:smoothing}).

Appendix \eqref{sec:appxStability} shows that the first (diffusive)
step of the implicit/explicit MPDATA $\theta$ scheme, eqn (\ref{eq:thetaUp}),
is stable and bounded when 
\begin{eqnarray}
\theta\ge1-\frac{1}{c} & \text{and} & \theta\ge0\label{eq:thetaBounds}
\end{eqnarray}
with the Courant number for an arbitrary mesh, $c$, defined as in
(\ref{eq:Courant}). Eqn (\ref{eq:thetaBounds}) can be used to set
$\theta_{f}$ based on the values of the Courant number in the cells
either side, $c_{\text{up}}$ and $c_{\text{down}}$ with a degree
of safety added to avoid reaching the stability limits:
\begin{equation}
\theta_{f}=\max\left\{ 1-\frac{1}{c_{\text{up}}+0.25},\ 1-\frac{1}{c_{\text{down}}+0.25},\ 0\right\} .
\end{equation}


\subsection{Additional Smoothing for Large Courant numbers \label{subsec:smoothing}}

Appendix \ref{sec:appxStability} shows that a linearised version
of the implicit/explicit MPDATA is unconditionally stable on a uniform,
one-dimensional grid. However this does not carry over onto an arbitrary
grid. Therefore $V_{f}$ is smoothed where $\theta>0$. First a cell
centre anti-diffusive flux is reconstructed from surrounding fluxes:
\begin{equation}
\mathbf{v}_{C}=\left(\sum_{f\in C}\mathbf{S}_{f}\mathbf{S}_{f}^{T}\right)^{-1}\sum_{f\in C}\mathbf{S}_{f}V_{f}
\end{equation}
which is the standard reconstruction of vectors from fluxes implemented
in OpenFOAM ($\sum_{f\in C}\mathbf{S}_{f}\mathbf{S}_{f}^{T}$ is a
tensor which can be inverted and pre-calculated for each cell). This
is a second-order accurate, least squares reconstruction which reconstructs
a uniform vector field exactly. The reconstructed velocity is then
interpolated back onto faces and the dot product taken with $\mathbf{S}_{f}$
to get a smoothed flux. The smoothed flux is used where $\theta>0$
and at the faces of cells with one face with $\theta>0$:
\begin{equation}
V_{f}=\begin{cases}
V_{f}\text{ from }(\ref{eq:Vcorr}) & \theta=0\\
\mathbf{v}_{Cf}\cdot\mathbf{S}_{f} & \theta>0
\end{cases}\label{eq:VcorrSmooth}
\end{equation}
where $\mathbf{v}_{Cf}$ is the reconstructed velocity $\mathbf{v}_{C}$
linearly interpolated from cell centres to faces.

\subsection{Linear Equation Solver}

The first-order upwind blended implicit/explicit advection creates
a sparse, asymmetric matrix $M$ with positive elements on the diagonal.
To create the matrix equation, (\ref{eq:thetaUp}) is re-arranged
so that the vector of new $\psi^{d}$ values ($\underline{\psi}^{d}$)
is a linear combination of old $\psi^{n}$ values ($\underline{\psi}^{n}$):
\begin{eqnarray}
M\underline{\psi}^{d} & = & N\underline{\psi}^{n}\\
\text{where }M_{ij} & = & \begin{cases}
1+\frac{\Delta t}{V_{i}}\sum_{f\in i}\theta_{f}\max\left\{ U_{f},0\right\}  & \text{for }i=j\\
-\frac{\Delta t}{V_{i}}\theta_{f}\max\left\{ -U_{f},0\right\}  & \text{where }f\text{ is between cells }i\text{ and }j
\end{cases}\\
\text{and }N_{ij} & = & \begin{cases}
1-\frac{\Delta t}{V_{i}}\sum_{f\in i}\left(1-\theta_{f}\right)\max\left\{ U_{f},0\right\}  & \text{for }i=j\\
+\frac{\Delta t}{V_{i}}\left(1-\theta_{f}\right)\max\left\{ -U_{f},0\right\}  & \text{where }f\text{ is between cells }i\text{ and }j
\end{cases}
\end{eqnarray}
Matrix $N$ is of course not created because the right hand vector
entries can be evaluated directly. If the flow is divergence free
then $\sum_{f\in i}U_{f}=0$ which implies that $M$ is strictly diagonally
dominant. $M$ will not be diagonally dominant at row $i$ if the
volume flux into cell $i$ in one time step is greater than the volume
flux out in that time step plus the cell volume. This situation is
not likely for atmospheric modelling as the atmosphere is low Mach
number but it would require either a smaller time step or a matrix
solver suitable for non-diagonally dominant matrices. 

The matrix equations generated are solved with the standard OpenFOAM
bi-conjugate gradient solver with a diagonal-based Incomplete LU preconditioner.
Details of tolerances and iteration counts is given in section \ref{subsec:SolverPerformace}.

\subsection{MPDATA Gauge}

The infinite gauge variant of MPDATA (equivalent to Lax-Wendroff)
can be used with implicit/explicit time stepping exactly as it is
used with standard, explicit MPDATA \citep{SC86,KS17}. This removes
the non-linearity of MPDATA and means that MPDATA no longer preserves
positivity. 

\subsection{Flux Corrected Transport (FCT) with Implicit Time Stepping\label{subsec:FCT}}

\citet{Zal79} state that their flux corrected transport (FCT) algorithm
can be used with implicit time stepping although we have not found
examples of this in the literature. In fact, the algorithm as described
by \citet{Zal79} does not guarantee monotonicity when used with implicit
time stepping. This is because \citet{Zal79} bound the tracer at
time $n+1$ by the diffusively transported tracer at time $n+1$ and
the tracer at time $n$ at the current and upwind grid points. The
tracer at time $n$ at the current and upwind grid points are not
a suitable bounds if the tracer can move a long distance in one time
step. When using implicit time stepping and large Courant numbers,
local extrema can be advected by more than one mesh cell in one time
step so local bounds from the previous time step no longer apply.
We therefore define two variants of FCT to work with implicit time
stepping. One guarantees monotonicity and the other guarantees boundedness
within pre-defined bounds. 

The first step if FCT is to advect using a monotonic, diffusive scheme
to calculate $\psi^{d}$. Appendix \ref{sec:appxBounded} shows that
the first-order upwind in space, blended explicit/implicit scheme
(eqn \ref{eq:thetaUp}) provides this solution for arbitrary Courant
numbers. The next step is to calculate the allowable minima and maxima
for each cell which we will call $\psi_{\min}$ and $\psi_{\max}$.
If we seek boundedness within pre-defined bounds then $\psi_{\min}$
and $\psi_{\max}$ are these bounds. Otherwise $\psi_{\min}$ and
$\psi_{\max}$ are the local extrema of $\psi^{d}$ in the current
and neighbouring cells. Explicit FCT also uses $\psi^{n}$ which widens
the bounds. Consequently, FCT for implicit, monotonic advection will
be more diffusive because of the use solely of $\psi^{d}$ to define
the local bounds:
\begin{eqnarray}
\text{for cell }C\ \psi_{\min} & = & \min_{N\in C}\left\{ \psi_{N}^{d}\right\} \ \text{where }N\ \text{are the face neighbours of }C\label{eq:boundsMin}\\
\text{for cell }C\ \psi_{\max} & = & \max_{N\in C}\left\{ \psi_{N}^{d}\right\} \ \text{where }N\ \text{are the face neighbours of }C.\label{eq:boundsMax}
\end{eqnarray}
We next define the maximum allowable amount that each cell can rise
or fall by and use the same notation as \citet{Zal79}
\begin{eqnarray}
Q_{p} & = & \psi_{\max}-\psi^{d}\\
Q_{m} & = & \psi^{d}-\psi_{\min}.
\end{eqnarray}
We next need to modify the unbounded MPDATA high order flux corrections
(HOC) from the implicit/explicit correction. This is the MPDATA flux
correction, $V_{f}$, from (\ref{eq:VcorrSmooth}) multiplied by $\psi_{d}$
at the upwind cell (upwind defined relative to $V_{f}$):
\begin{equation}
F_{f\text{HOC}}=\psi_{d\text{up}}V_{f}.
\end{equation}
From this we calculate the total high order flux that enters ($P_{p}$)
and leaves ($P_{m}$) each cell:
\begin{eqnarray}
P_{p} & = & -\frac{\Delta t}{V_{C}}\sum_{f\in C}\min\left\{ F_{f\text{HOC}},\ 0\right\} \\
P_{m} & = & \frac{\Delta t}{V_{C}}\sum_{f\in C}\max\left\{ F_{f\text{HOC}},\ 0\right\} .
\end{eqnarray}
Next we find the ratios of the allowable total fluxes to the actual
high order fluxes:
\begin{eqnarray}
R_{p} & = & \begin{cases}
\min\left\{ 1,\ \frac{Q_{p}}{P_{p}}\right\}  & \text{if }P_{p}>0\\
0 & \text{otherwise}
\end{cases}\\
R_{m} & = & \begin{cases}
\min\left\{ 1,\ \frac{Q_{m}}{P_{m}}\right\}  & \text{if }P_{m}>0\\
0 & \text{otherwise.}
\end{cases}
\end{eqnarray}
Finally we find the coefficient to multiply $F_{f\text{HOC}}$ in
order to achieve either a monotonic solution or a solution with the
required bounds:
\begin{eqnarray*}
F_{f} & = & \begin{cases}
F_{f\text{HOC}}\min\left\{ R_{pN},\ R_{mC}\right\}  & \text{if }F_{f\text{HOC}}\ge0\\
F_{f\text{HOC}}\min\left\{ R_{pC},\ R_{mN}\right\}  & \text{otherwise}
\end{cases}\\
\text{where} &  & \left(\mathbf{x}_{N}-\mathbf{x}_{C}\right)\cdot\mathbf{S}_{f}>0
\end{eqnarray*}
and cells $C$ and $N$ are either side of face $f$. Then the final
update is
\begin{equation}
\psi_{C}^{n+1}=\psi_{C}^{d}-\frac{\Delta t}{V_{C}}\sum_{f\in C}F_{f}
\end{equation}
which is monotonic if (\ref{eq:boundsMin}) and (\ref{eq:boundsMax})
are used to bound $\psi$ and bounded if specific bounds are specified.

\section{Advection Test Cases}

\subsection{One-dimensional Advection}

\begin{figure}
\begin{centering}
Implicit/explicit MPDATA
\par\end{centering}
\begin{centering}
\includegraphics[width=0.75\textwidth]{/home/hilary/OpenFOAM/hilary-7/run/hilary/advection/1d/nonUniform/twoShapes/MPDATA/mesh10_100/1/Tcompare}
\par\end{centering}
\begin{centering}
Implicit/explicit infinite gauge MPDATA
\par\end{centering}
\begin{centering}
\includegraphics[width=0.75\textwidth]{/home/hilary/OpenFOAM/hilary-7/run/hilary/advection/1d/nonUniform/twoShapes/LW/mesh10_100/1/Tcompare}
\par\end{centering}
\begin{centering}
Implicit/explicit infinite gauge MPDATA with FCT
\par\end{centering}
\begin{centering}
\includegraphics[width=0.75\textwidth]{/home/hilary/OpenFOAM/hilary-7/run/hilary/advection/1d/nonUniform/twoShapes/LW_FCT/mesh10_100/1/Tcompare}
\par\end{centering}
\caption{Advection once around a periodic domain starting from mixed initial
conditions $\psi^{0}$ using 40 grid points for the uniform resolution
and 100 grid points for the resolution with a factor $R=10$ between
finest and coarsest. The regions where the non-uniform resolution
has a Courant number greater than 0.75 (where implicit time stepping
is used) is shaded grey.\label{fig:1dresults}}
\end{figure}

The first test of the blended implicit/explicit MPDATA is in one dimensional
with uniform velocity. In order to vary the Courant number in space
so that the scheme is implicit and explicit in different places, variable
resolution is used and compared with uniform resolution simulations.
The variable resolution grids have resolution a factor of $R=10$
finer in the middle of the unit length domain than the end points.
There are $n$ cells ($n+1$ grid points) in the unit length and a
constant ratio, $r=R^{\frac{2}{n-2}}$, between successive cells in
the first half of the domain and $\frac{1}{r}$ in the second half.
Therefore the resolution of cell $i$ is:
\begin{equation}
\Delta x_{i}=\begin{cases}
\frac{1}{2}Rr^{-i}\frac{1-r}{1-rR} & i\le\frac{n}{2}-1\\
\frac{1}{2}Rr^{\frac{n}{2}-i}\frac{1-r}{1-rR} & i\ge\frac{n}{2}.
\end{cases}
\end{equation}
We use smooth initial conditions for evaluating convergence with resolution
and mixed initial conditions for inspecting boundedness and overall
quality of solution:
\begin{eqnarray}
\psi_{\text{smooth}}^{0} & = & \begin{cases}
\frac{1}{2}\left\{ 1+\cos\pi\left(4x-1\right)\right\}  & x\in\left[0,0.5\right]\\
0 & \text{otherwise}
\end{cases}\\
\psi_{\text{mixed}}^{0} & = & \begin{cases}
\frac{1}{2}\left\{ 1+\cos\pi\left(4x-1\right)\right\}  & x\in\left[0,0.5\right]\\
1 & x\in\left[0.6,0.8\right]\\
0 & \text{otherwise.}
\end{cases}
\end{eqnarray}
All simulations use velocity of $u=1$ and run for one time unit so
that the tracer travels one complete revolution around the periodic
domain.

Figure \ref{fig:1dresults} shows solutions starting from the mixed
initial conditions using 100 time steps each of length $\Delta t=0.01$.
The uniform resolution has 40 cells giving a uniform Courant number
of 0.4 (meaning that the time stepping is purely explicit). The non-uniform
resolution has 100 cells with $R=10$ giving a Courant numbers in
the range $c\in\left[0.4,4\right]$ so that implicit time stepping
is used where $c>0.75$. As expected, the MPDATA results (top row
of figure \ref{fig:1dresults}) are always positive for both the uniform
resolution (explicit time stepping) and the non-uniform resolution
(implicit/explicit time stepping). The non-uniform resolution solution
produces a stable overshoot above the square wave which cannot be
ruled out by MPDATA without limited fluxes. The infinite gauge version
of MPDATA (middle row of figure \ref{fig:1dresults}) produces undershoots
and overshoots and the solution is more accurate in the region of
the smooth wave. Neither the uniform (explicit) or non-uniform (implicit/explicit)
results appear more accurate than the other. The results using flux-corrected
transport (FCT, bottom row of figure \ref{fig:1dresults}) are bounded
above and below, demonstrating the efficacy of the flux corrections
applied to implicit/explicit time stepping. The uniform (explicit)
or non-uniform (implicit/explicit) results appear very similar.

\noindent 
\begin{figure}
\begin{centering}
\includegraphics[width=0.75\textwidth]{/home/hilary/OpenFOAM/hilary-7/run/hilary/advection/1d/nonUniform/cosBell/plots/l2converge}
\par\end{centering}
\caption{Convergence of the $\ell_{2}$ error norm with resolution of the one-dimensional
advection of the smooth initial conditions, $\psi^{0}$. The uniform
resolutions use 20, 40 and 80 grid points and the non-uniform resolutions
use 50, 100 and 200 grid points with ratio $R=10$. Both use time
steps of $\Delta t=0.02$, 0.01 and 0.005. Dotted lines show the slope
of first and second-order convergence. \label{fig:1d_converge}}
\end{figure}

Convergence with resolution for all schemes on uniform and non-uniform
meshes starting from the smooth initial conditions is shown in figure
\ref{fig:1d_converge}. The uniform meshes always use purely explicit
time stepping ($c=0.4$) and the non-uniform meshes use blended implicit/explicit
since $c\in\left[0.4,4\right]$. MPDATA with and without the infinite
gauge and with and without FCT give second-order convergence. Even
though the non-uniform grid means that the Courant number reaches
4 at the centre of the domain, the convergence remains strong.

\subsection{Spherical Meshes}

Advection test cases using implicit/explicit MPDATA calculated using
various meshes are presented. There is no clearly optimal mesh of
the sphere for atmospheric modelling (example meshes in fig. \ref{fig:grids}).
Numerical methods need to be designed to allow for one or more of
the following features of meshes of the sphere:
\begin{enumerate}
\item Latitude-longitude meshes are orthogonal and have uniform resolution
following co-ordinate lines but they have severe convergence of mesh
lines towards two poles so numerical methods are needed that can cope
with very large Courant numbers. We use a latitude-longitude mesh
with a cell at each pole (fig. \ref{fig:grids}).
\item Hexagonal and triangular meshes of the sphere are quasi-uniform but
they cannot be all three of:
\begin{enumerate}
\item orthogonal (mesh lines and cell centre to cell centre lines cross
at right angles)
\item centroidal (cell centres are at cell centroids)
\item non-skew (cell centre to cell centre lines bisect mesh lines)
\end{enumerate}
meaning that special numerical treatment is needed in order to achieve
second-order accuracy.
\item Quasi-uniform versions of the cubed-sphere are non-orthogonal with
large distortions (skewness) at cube edges and corners so numerical
methods are needed that maintain accuracy at these distortions. The
cubed-sphere in fig. \ref{fig:grids} uses the Gnomonic projection
\citep{RPM96}.
\item Skipped latitude-longitude meshes have factor of two reductions in
resolution in the longitudinal direction at a few latitudes to prevent
the mesh lines converging. At latitudes where the resolution reduces
the meshes can be treated as non-conforming so that two quadrilateral
cells are connected to one edge of the adjacent quadrilateral cell,
or conforming with two quadrilaterals connected to adjacent, aligned
edges of a distorted pentagon. The implementation described here treats
them as conforming.
\end{enumerate}
\begin{figure}
\begin{tabular}{cc}
\includegraphics[width=0.48\textwidth]{/home/hilary/OpenFOAM/hilary-7/run/hilary/advection/deformingOnSphere/fullDeformation/MPDATA_latLonPolarSkipped_c2/latLon_48x24/constant/mesh} & \includegraphics[width=0.48\textwidth]{/home/hilary/OpenFOAM/hilary-7/run/hilary/advection/deformingOnSphere/weakDeformation/c1_MPDATA_cubedSphere_2gauge_adv/cube_15/constant/mesh}\tabularnewline
Skipped latitude-longitude, $48\times24$, 818 cells & Gnomonic cubed-sphere, $15\times15\times6=1,350$ cells \tabularnewline
\includegraphics[width=0.48\textwidth]{/home/hilary/OpenFOAM/hilary-7/run/hilary/advection/deformingOnSphere/weakDeformation/MPDATA_HRgrid_2gauge_adv/HRgrid4/constant/mesh} & \includegraphics[width=0.48\textwidth]{/home/hilary/OpenFOAM/hilary-7/run/hilary/advection/deformingOnSphere/weakDeformation/MPDATA_tri_2gauge_adv/tri4/constant/mesh}\tabularnewline
Hexagaonal icosahedron, 642 cells & Triangular icosahedron, 1280 cells\tabularnewline
\end{tabular}

\caption{Some common meshes of the sphere viewed from above a point at a latitude
of $45^{o}$.\label{fig:grids}}
\end{figure}

All of the meshes are decomposed into four domains for parallel processing
with MPI.

\subsection{Deformational Flow}

\citet{LSPT12} describe deformational flow test cases to demonstrate
a number of numerical properties of an advection scheme including
order of convergence and monotonicity. We are using the non-divergent
wind field which deforms and translates the initial conditions so
that the final solution (at $t=T=5$) should be identical to the initial
conditions (at $t=0$). The wind is defined by a stream function,
$\Psi$, based on latitude, $\phi$, longitude, $\lambda$, time,
$t$ and the radius of the sphere, $R=1$:
\begin{equation}
\Psi\left(\lambda,\phi,t\right)=\frac{10R}{T}\sin^{2}\left(\lambda-\frac{2\pi t}{T}\right)\cos^{2}\phi\cos\frac{\pi t}{T}-\frac{2\pi R}{T}\sin\phi
\end{equation}


\subsubsection{Gaussian Hills}

The Gaussian hills initial conditions are infinitely smooth and so
can be used to measure the numerical order of convergence. The initial
conditions of the tracer, $\psi_{0}$, are given in terms of the three
dimensional position vector, $\mathbf{x}$, in Caresian co-ordinates:
\begin{eqnarray}
\psi_{0}\left(\mathbf{x}\right) & = & 0.95\left[\exp\left\{ -5\left(\mathbf{x}-\mathbf{x}_{1}\right)^{2}\right\} +\exp\left\{ -5\left(\mathbf{x}-\mathbf{x}_{2}\right)^{2}\right\} \right]\\
\text{where }\mathbf{x}_{i} & = & \left(R\cos\phi_{i}\cos\lambda_{i},\ R\cos\phi_{i}\sin\lambda_{i},\ R\sin\phi_{i}\right)\\
\left(\lambda_{1},\phi_{1}\right) & = & \left(5\pi/6,\ 0\right)\\
\left(\lambda_{2},\phi_{2}\right) & = & \left(7\pi/6,\ 0\right)
\end{eqnarray}
The tracer concentrations at $t=2.5$ are show in fig. \ref{fig:fullDeformation2p5}
calculated on five different meshes of the sphere and for a $30^{o}$
rotated version of the latitude-longitude mesh, all at a similar resolution.
The implicit/explicit $\theta$ MPDATA is used without flux limiting
and without a gauge.

Simulations using all the meshes in fig. \ref{fig:fullDeformation2p5}
use a time step of $0.01$ (500 time steps in total) giving a Courant
number of around 2 so that the simulations would be unstable if a
purely explicit scheme were used (although at $t=2.5$ on the latitude-longitude
mesh the Courant number is everywhere below 0.8). Resolutions and
time steps are shown in table \ref{tab:dx_dt}.
\noindent \begin{center}
\begin{table}
\noindent \begin{centering}
\begin{tabular}{|l|c|c|c|c|c|}
\hline 
Mesh type & Nominal & N. cells & $\Delta x$ & $\Delta t$ & Figure\tabularnewline
\hline 
\hline 
Latitude-longitude & $120\times60$ & 7,080 & $3.0^{o}$ & 0.02 & \ref{fig:MPDATAGuassiandiagnostics}c, \ref{fig:MPDATAGaugeDiagnostics}b\tabularnewline
\hline 
 & $240\times120$ & 28,800 & $1.5^{o}$ & 0.01 & \ref{fig:fullDeformation2p5}, \ref{fig:MPDATAGuassiandiagnostics},
\ref{fig:MPDATAGaugeDiagnostics}\tabularnewline
\hline 
 & $480\times240$ & 114,720 & $0.75^{o}$ & 0.005 & \ref{fig:MPDATAGuassiandiagnostics}c, \ref{fig:MPDATAGaugeDiagnostics}b,\ref{fig:fullDeformationSlot5}\tabularnewline
\hline 
\hline 
Skipped latitude- & $48\times24$ & 864 & $7.5^{o}$ &  & \ref{fig:grids}\tabularnewline
\hline 
 & $120\times60$ & 5,310 & $3.0^{o}$ & 0.02 & \ref{fig:MPDATAGuassiandiagnostics}c, \ref{fig:MPDATAGaugeDiagnostics}b\tabularnewline
\hline 
longitude & $240\times120$ & 21,750 & $1.5^{o}$ & 0.01 & \ref{fig:fullDeformation2p5}, \ref{fig:MPDATAGuassiandiagnostics},
\ref{fig:MPDATAGaugeDiagnostics}\tabularnewline
\hline 
 & $480\times240$ & 88,470 & $0.75^{o}$ & 0.005 & \ref{fig:MPDATAGuassiandiagnostics}c, \ref{fig:MPDATAGaugeDiagnostics}b,\ref{fig:fullDeformationSlot5}\tabularnewline
\hline 
\hline 
Cubed-sphere & $15\times15\times6$ & 1,350 & $6.4^{o}$ &  & \ref{fig:grids}\tabularnewline
\hline 
 & $30\times30\times6$ & 5,400 & $3.2^{o}$ & 0.02 & \ref{fig:MPDATAGuassiandiagnostics}c, \ref{fig:MPDATAGaugeDiagnostics}b\tabularnewline
\hline 
 & $60\times60\times6$ & 21,600 & $1.6^{o}$ & 0.01 & \ref{fig:fullDeformation2p5}, \ref{fig:MPDATAGuassiandiagnostics},
\ref{fig:MPDATAGaugeDiagnostics}\tabularnewline
\hline 
 & $120\times120\times6$ & 86,400 & $0.8^{o}$ & 0.005 & \ref{fig:MPDATAGuassiandiagnostics}c, \ref{fig:MPDATAGaugeDiagnostics}b,\ref{fig:fullDeformationSlot5}\tabularnewline
\hline 
\hline 
Hexagonal- & HR4 & 642 & $9.5^{o}$ &  & \ref{fig:grids}\tabularnewline
\hline 
 & HR6 & 10,242 & $2.4^{o}$ & 0.02 & \ref{fig:MPDATAGuassiandiagnostics}c, \ref{fig:MPDATAGaugeDiagnostics}b\tabularnewline
\hline 
icosahedral & HR7 & 40,962 & $1.2^{o}$ & 0.01 & \ref{fig:fullDeformation2p5}, \ref{fig:MPDATAGuassiandiagnostics},
\ref{fig:MPDATAGaugeDiagnostics}\tabularnewline
\hline 
 & HR8 & 163,842 & $0.6^{o}$ & 0.005 & \ref{fig:MPDATAGuassiandiagnostics}c, \ref{fig:MPDATAGaugeDiagnostics}b,\ref{fig:fullDeformationSlot5}\tabularnewline
\hline 
\hline 
Triangular- & Tri4 & 1,280 & $5.4^{o}$ &  & \ref{fig:grids}\tabularnewline
\hline 
 & Tri6 & 20,480 & $1.4^{o}$ & 0.01 & \ref{fig:fullDeformation2p5}, \ref{fig:MPDATAGuassiandiagnostics},
\ref{fig:MPDATAGaugeDiagnostics}\tabularnewline
\hline 
icosahedral & Tri7 & 81,920 & $0.68^{o}$ & 0.005 & \ref{fig:MPDATAGuassiandiagnostics}c, \ref{fig:MPDATAGaugeDiagnostics}b,\ref{fig:fullDeformationSlot5}\tabularnewline
\hline 
 & Tri8 & 327,680 & $0.34^{o}$ & 0.0025 & \ref{fig:MPDATAGuassiandiagnostics}c, \ref{fig:MPDATAGaugeDiagnostics}b\tabularnewline
\hline 
\end{tabular}
\par\end{centering}
\caption{Resolutions and time steps for deformational advection.\label{tab:dx_dt}}
\end{table}
\par\end{center}

The flow goes to zero at the north and south poles so the convergence
of meridians of the un-rotated latitude-longitude mesh does not lead
to large Courant numbers. However when the mesh is rotated by $30^{o}$,
high winds cross the poles of the mesh so the maximum Courant number
goes up to 70 (the contours in figure \ref{fig:fullDeformation2p5}
show the Courant number). These large Courant numbers do not lead
to instability, a lack of positivity or visible artefacts in the solution.
The large Courant numbers are removed on the rotated, skipped latitude
longitude mesh although some values above 0.8 are visible at $t=2.5$
just poleward of the change in longitudinal resolution. On the cubed-sphere,
the Courant number is largest near the cube corners due to mesh distortions
and smaller cell areas. Some mesh imprinting is visible along the
cube edges although this does not lead to a lack of positivity. The
hexagonal and triangular icosahedral meshes are the most uniform meshes
of the sphere and so no locally high Courant numbers are visible at
$t=2.5$. The results from the hexagonal mesh appear accurate but
note that this mesh has higher resolution that the other meshes, which
explains the higher Courant number than the triangular mesh. On the
triangular mesh, the maximum values are increased (from 0.95 to 1.08
at 2.5s on triangular mesh 6). The anti-diffusion appears too strong
which could be dangerous as it could lead to instability. It would
probably be possible to optimise MPDATA for the triangular mesh but
this has not been done. 

\begin{figure}
\begin{tabular}{cc}
\includegraphics[width=0.48\textwidth]{/home/hilary/OpenFOAM/hilary-7/run/hilary/advection/deformingOnSphere/fullDeformation/MPDATA_latLon_c2/latLon_240x120/2.5/Traw} & \includegraphics[width=0.48\textwidth]{/home/hilary/OpenFOAM/hilary-7/run/hilary/advection/deformingOnSphere/fullDeformation/MPDATA_latLonRotated_c2/latLon_240x120/2.5/Traw}\tabularnewline
Full lat-lon, $240\times120$, $\Delta x=1.5^{o}$ & Rotated, full lat-lon, 30$^{o}$, $240\times120$, $\Delta x=1.5^{o}$\tabularnewline
28,800 cells, $\Delta t=0.01$ , $c\le2$ & 28,800 cells, $\Delta t=0.01$ , $c\le70$\tabularnewline
\includegraphics[width=0.48\textwidth]{/home/hilary/OpenFOAM/hilary-7/run/hilary/advection/deformingOnSphere/fullDeformation/MPDATA_latLonSkipped_c2/latLon_240x120/2.5/Traw} & \includegraphics[width=0.48\textwidth]{/home/hilary/OpenFOAM/hilary-7/run/hilary/advection/deformingOnSphere/fullDeformation/MPDATA_cubedSphere_c2/cube_60/2.5/Traw}\tabularnewline
Skipped, rotated lat-lon, $240\times120$, $\Delta x\ge1.5^{o}$ & Cubed-sphere, $60\times60\times6$, $\Delta x\sim1.5^{o}$\tabularnewline
21,750 cells, $\Delta t=0.01$ , $c<4.5$ & 21,600 cells, $\Delta t=0.01$ , $c<3.1$\tabularnewline
\includegraphics[width=0.48\textwidth]{/home/hilary/OpenFOAM/hilary-7/run/hilary/advection/deformingOnSphere/fullDeformation/MPDATA_HRgrid_c2/HRgrid7/2.5/Traw} & \includegraphics[width=0.48\textwidth]{/home/hilary/OpenFOAM/hilary-7/run/hilary/advection/deformingOnSphere/fullDeformation/MPDATA_tri_c2/tri6/2.5/Traw}\tabularnewline
Hexagaonal mesh 7, $\Delta x\sim1.2^{o}$ & Triangular mesh 6, $\Delta x\sim1.3^{o}$\tabularnewline
40,962 cells, $\Delta t=0.01$ , $c\le2$ & 20,480 cells, $\Delta t=0.01$ , $c<2$\tabularnewline
\multicolumn{2}{c}{\includegraphics[width=0.48\textwidth]{/home/hilary/OpenFOAM/hilary-7/run/hilary/advection/deformingOnSphere/legends/T}}\tabularnewline
\end{tabular}

\caption{Deformational flow on the sphere after 2.5 time units. The colours
show the tracer (piece-wise constant in each cell). The contours show
the Courant number contoured every 0.2 starting from $c=0.8$ (grey)
to $c=2$ (black) and dashed contours every 1 for $c\ge3$. \label{fig:fullDeformation2p5}}
\end{figure}

MPDATA is, by design, positivity preserving but not bounded above
or monotonic. The implicit/explicit MPDATA retains this feature on
all of the meshes tested and displayed in figure \ref{fig:fullDeformation2p5}.
The mininum and maximum tracer values for all time steps for each
of the meshes in figure \ref{fig:fullDeformation2p5} are shown at
the top of fig. \ref{fig:MPDATAGuassiandiagnostics}. All of the minima
remain positive and very close to zero. The maxima decrease due to
numerical diffusion but they do not decrease monotonically MPDATA
is not a monotonic scheme. The maximum on the triangular mesh increases
and oscillates before it decreases, would could be a feature that
makes it prone to instability.

\begin{figure}
Maximum and minimum tracer values for all time steps

\includegraphics[width=0.8\textwidth]{/home/hilary/OpenFOAM/hilary-7/run/hilary/advection/deformingOnSphere/plots/TminMax}

Maximum and mean (dashed) Courant number for all time steps

\includegraphics[width=0.8\textwidth]{/home/hilary/OpenFOAM/hilary-7/run/hilary/advection/deformingOnSphere/plots/CoMaxMean}

Convergence of $\ell_{2}$ error norm with resolution

\includegraphics[width=0.85\textwidth]{/home/hilary/OpenFOAM/hilary-7/run/hilary/advection/deformingOnSphere/plots/l2error}

\caption{Diagnostics of the results for the deformational flow of the Gaussian
hills with standard implicit/explicit MPDATA (no flux corrected transport
and no gauge). Top and middle are diagnostics of the simulations shown
in figure \ref{fig:fullDeformation2p5}. Bottom includes other resolutions.
Mesh and time step details in table \ref{tab:dx_dt}.\label{fig:MPDATAGuassiandiagnostics}}
\end{figure}

The maximum and mean Courant number for each time step for each of
the meshes in figure \ref{fig:fullDeformation2p5} are shown in the
middle row of fig. \ref{fig:MPDATAGuassiandiagnostics}. The maximum
Courant number for all meshes is greater than one using the time step
of 0.01 and is minimum at the middle of the simulation ($t=2.5$).
The maximum Courant number for the rotated latitude-longitude mesh
reaches 70 and is always much larger than one which does not appear
to significantly reduce the accuracy. The mean Courant numbers (dashed)
are below or close to one throughout which helps to maintain accuracy.

The convergence of the $\ell_{2}$ error norm with resolution is shown
in the bottom row of fig. \ref{fig:MPDATAGuassiandiagnostics}. The
mesh resolutions and time steps for these simulations are given in
table \ref{tab:dx_dt}. The resolution to time step ratio is kept
constant along each line. Included in this graph is simulations using
half the time step and five times the time step for the latitude-longitude
mesh (giving maximum Courant number around one and around ten in order
show the impact of varying the mean Courant number. Reducing the time
step to get $c<1$ means that the standard explicit MPDATA is used
almost everywhere. Surprisingly, this increases the error slightly.
When the Courant number is close to 2, $\theta$ is close to $\frac{1}{2}$
and the temporal error correction is small. This implies that using
second-order implicit/explicit time stepping is more accurate than
using first-order time stepping with a correction. However the implicit/explicit
time stepping requires a matrix inversion and so is more expensive.
The simulation with $c<10$ is much less accurate because the temporal
correction is not applied for $c>2$. However the simulation is still
stable and positive definite.

The convergence with resolution in fig. \ref{fig:MPDATAGuassiandiagnostics}
is around first-order at coarse resolution and approaches second-order
at higher resolution. It is expected for second-order advection schemes
to converge at less that second order at coarse resolution for this
test case, as demonstrated in figures 1 and 2 of \citet{LUJ+14} who
presented results of this test case for a variety of advection schemes
on different meshes. Errors saturate at coarse resolution and so the
theoretical convergence rate is not achieved. The errors in figure
\ref{fig:MPDATAGuassiandiagnostics} are similar to the second-order
schemes presented in \citet{LUJ+14}. The exception is on the triangular
mesh which looses accuracy at the highest resolution due to instabilities. 

Better accuracy at the expense of positivity can be achieved with
MPDATA by using the infinite gauge variant (equivalent to Lax-Wendroff)
which works for the $\theta$ implicit/explicit version in the same
way as the standard MPDATA \citep{SC86}. The maximum and minimum
values of the tracer for infinite gauge simulations with the same
resolution as those shown in figure \ref{fig:fullDeformation2p5}
are shown in figure \ref{fig:MPDATAGaugeDiagnostics}. In comparison
to the MPDATA simulations without a gauge (figure \ref{fig:MPDATAGuassiandiagnostics})
the gauge results have a smaller reduction in the maximum (because
the results are more accurate and hence less diffusive) but the minima
is less than zero (spurious undershoots are generated). $\ell_{2}$
errors with resolution are shown at the bottom of fig. \ref{fig:MPDATAGaugeDiagnostics}.
The mesh spacing and times steps are the same as in figure \ref{fig:MPDATAGuassiandiagnostics}
and are shown in table \ref{tab:dx_dt} . These simulations use one
correction per time step. The order of convergence is higher and the
$\ell_{2}$ errors lower than MPDATA results without a gauge (figure
\ref{fig:MPDATAGuassiandiagnostics}). Again the results on the triangular
mesh are not well behaved which could probably be fixed with optimisation
for a triangular mesh.

\begin{figure}
Maximum and minimum tracer values for all time steps

\includegraphics[width=0.8\textwidth]{/home/hilary/OpenFOAM/hilary-7/run/hilary/advection/deformingOnSphere/plots/TminMaxGauge}

Convergence of $\ell_{2}$ error norm with resolution

\includegraphics[width=0.8\textwidth]{/home/hilary/OpenFOAM/hilary-7/run/hilary/advection/deformingOnSphere/plots/l2errorGauge}

\caption{Diagnostics of the results for the deformational flow of the Gaussian
hills with implicit/explicit MPDATA with no flux corrected transport
but with a gauge of 10. Other settings the same as fig. \ref{fig:MPDATAGuassiandiagnostics}.
\label{fig:MPDATAGaugeDiagnostics}}
\end{figure}


\subsubsection{Slotted Cylinders}

Deformational advection of slotted cylinders tests the implementation
of limiters. \citet{LSPT12} recommend the same deformational velocity
field as for the Gaussian hills with initial tracers defined by:
\begin{eqnarray}
\psi_{0}\left(\lambda,\phi\right) & = & \begin{cases}
1 & \text{if }r_{i}\le r\text{ and }|\lambda-\lambda_{i}|\ge\frac{r}{6R}\text{ for }i=1,2\\
1 & \text{if }r_{1}\le r\text{ and }|\lambda-\lambda_{1}|<\frac{r}{6R}\text{ and }\phi-\phi_{1}<-\frac{5}{12}\frac{r}{R}\\
1 & \text{if }r_{2}\le r\text{ and }|\lambda-\lambda_{2}|<\frac{r}{6R}\text{ and }\phi-\phi_{2}>\frac{5}{12}\frac{r}{R}\\
0.1 & \text{otherwise}
\end{cases}\\
\text{where }\mathbf{x} & = & \left(R\cos\phi\cos\lambda,\ R\cos\phi\sin\lambda,\ R\sin\phi\right)\\
r & = & R/2,\ r_{i}=|\mathbf{x}-\mathbf{x}_{i}|\\
\left(\lambda_{1},\phi_{1}\right) & = & \left(5\pi/6,\ 0\right)\\
\left(\lambda_{2},\phi_{2}\right) & = & \left(7\pi/6,\ 0\right).
\end{eqnarray}
The tracer fields at the end of the simulations ($t=T=5$) are shown
in fig. \ref{fig:fullDeformationSlot5} for all meshes at the highest
resolution used and at time steps giving Courant numbers of around
2 (see table \ref{tab:dx_dt}). This uses the implicit/explicit infinite
guage MPDATA and flux corrected transport limited to ensure monotonicity.
Fig. \ref{fig:fullDeformationSlot5} shows that the bounds of the
initial conditions are maintained and no new extrema are generated,
even on the rotated latitude-longitude mesh where the Courant number
reaches 140. The skipped latitude-longitude mesh has sharp jumps in
the Courant number which do not cause artefacts in the solution. The
flux correction also removes the overshoots from the triangular mesh
solution. This is, to our knowledge, the first monotonic and conservative
solution of the advection equation using such a large Courant number.

\noindent 
\begin{figure}
\begin{tabular}{cc}
\includegraphics[width=0.48\textwidth]{/home/hilary/OpenFOAM/hilary-7/run/hilary/advection/deformingOnSphere/slottedCylinder/LW_FCT_latLonPolar_c2/latLon_480x240/5/TslotRaw} & \includegraphics[width=0.48\textwidth]{/home/hilary/OpenFOAM/hilary-7/run/hilary/advection/deformingOnSphere/slottedCylinder/LW_FCT_latLonPolarRotated_c2/latLon_480x240/5/TslotRaw}\tabularnewline
Full lat-lon, $480\times240$, $\Delta x=0.75{}^{o}$ & Rotated 30$^{o}$, full lat-lon, $480\times240$, $\Delta x=0.75{}^{o}$\tabularnewline
115,200 cells, $\Delta t=0.005$ , $c\le2$ & 115,200 cells, $\Delta t=0.005$ , $c\le140$\tabularnewline
\includegraphics[width=0.48\textwidth]{/home/hilary/OpenFOAM/hilary-7/run/hilary/advection/deformingOnSphere/slottedCylinder/LW_FCT_latLonPolarSkipped_c2/latLon_480x240/5/TslotRaw} & \includegraphics[width=0.48\textwidth]{/home/hilary/OpenFOAM/hilary-7/run/hilary/advection/deformingOnSphere/slottedCylinder/LW_FCT_cubedSphere_c2/cube_120/5/TslotRaw}\tabularnewline
Skipped, rotated lat-lon, $480\times240$, $\Delta x\ge0.75^{o}$ & Cubed-sphere, $120\times120\times6$, $\Delta x\sim0.75^{o}$\tabularnewline
88,470 cells, $\Delta t=0.005$ , $c<4.5$ & 86,400 cells, $\Delta t=0.005$ , $c<3.1$\tabularnewline
\includegraphics[width=0.48\textwidth]{/home/hilary/OpenFOAM/hilary-7/run/hilary/advection/deformingOnSphere/slottedCylinder/LW_FCT_HRgrid_c2/HRgrid8/5/TslotRaw} & \includegraphics[width=0.48\textwidth]{/home/hilary/OpenFOAM/hilary-7/run/hilary/advection/deformingOnSphere/slottedCylinder/LW_FCT_tri_c2/tri7/5/TslotRaw}\tabularnewline
Hexagaonal mesh 8, $\Delta x\sim0.6^{o}$ & Triangular mesh 7, $\Delta x\sim0.7^{o}$\tabularnewline
163,842 cells, $\Delta t=0.005$ , $c\le2$ & 81,920 cells, $\Delta t=0.005$ , $c<2$\tabularnewline
\multicolumn{2}{c}{\includegraphics[width=0.48\textwidth]{/home/hilary/OpenFOAM/hilary-7/run/hilary/advection/deformingOnSphere/legends/TslotRaw_T}}\tabularnewline
\end{tabular}

\caption{Deformational flow on the sphere after 5 time units. The colours show
the piecewise uniform value of the tracer value in each cell. The
contours show the Courant number contoured every 0.2 starting from
$c=0.8$ (grey) to $c=2$ (black) and dotted black contours every
1 for $c\ge3$. \label{fig:fullDeformationSlot5}}
\end{figure}


\subsection{Solver Performance\label{subsec:SolverPerformace}}

Solver performance is reported for a selection of simulations using
full latitude-longitude meshes as the large inhomogeneity of cell
size and large range of Courant numbers leads to an ill conditioned
matrix and poor solver performance \citep{TB15}. The full latitude-longitude
mesh therefore represents a worst case. Each time step consists of
one implicit solve using the standard OpenFOAM bi-conjugate gradient
solver with a diagonal-based Incomplete LU preconditioner. 

\begin{figure}
\includegraphics[width=1\textwidth]{/home/hilary/OpenFOAM/hilary-7/run/hilary/advection/deformingOnSphere/plots/iterations}

\caption{Number of solver iterations per time step for simulations on unrotated
latitude-longitude meshes. The legend shows if the simulation used
a solver tolerance of $10^{-6}$ or $10^{-12}$. \label{fig:nIterations}}
\end{figure}

The number of iterations of the solver per time step is shown in fig.
\ref{fig:nIterations} for various simulations on different resolutions
of the latitude-longitude mesh. The number of iterations is smallest
around time 2.5 when the wind speed is lowest and so the Courant number
is smallest. For simulations with the maximum Courant number less
than 0.75 in the middle of the simulations, the number of solver iterations
drops to zero because the simulation is purely explicit. The simulations
represented by black lines have a maximum Courant number of 1 and
so the number of iterations is small throughout the simulation. For
the simulations with a maximum Courant number of 2 (in blue), the
number of iterations is always below 5. Once the maximum Courant number
reaches 10, the number of iterations can be over 15. The number of
iterations increases slower than linearly with Courant number, which
is necessary for efficiency. However it is possible to reduce the
number of iterations by reducing the solver tolerance. The tighter
tolerance made less than 1\% difference to the error norms of the
final solution (not shown). Higher spatial resolution has little influence
on the number of iterations per time step. 

\section{Conclusions}

\section*{Acknowledgements}

\bibliographystyle{abbrvnat}
\bibliography{numerics}


\appendix

\section{Boundedness of the first-order Upwind Implicit/Explicit Scheme\label{sec:appxBounded}}

The first-order upwind, implicit/explicit scheme can be written:
\begin{eqnarray}
\psi_{C}^{n+1}=\psi_{C}^{n} & + & \frac{\Delta t}{V_{C}}\sum_{i\in\text{in}}\left(1-\theta_{i}\right)U_{i}\psi_{i}^{n}+\frac{\Delta t}{V_{C}}\sum_{i\in\text{in}}\theta_{i}U_{i}\psi_{i}^{n+1}\label{eq:thetaUp-2}\\
 & - & \frac{\Delta t}{V_{C}}\sum_{o\in\text{out}}\left(1-\theta_{o}\right)U_{o}\psi_{C}^{n}-\frac{\Delta t}{V_{C}}\sum_{o\in\text{out}}\theta_{o}U_{o}\psi_{C}^{n+1}\nonumber 
\end{eqnarray}
for cell $C$ with faces $i\in\text{in}$ with flow into cell $C$
and faces $j\in\text{out}$ with flow out of cell $C$, off-centering
values $\theta_{i,j}\in[0,1]$, $U_{i,o}$ the volume fluxes through
faces $i,o$ and $\psi_{i}$ the values of $\psi$ in cells through
the $i$ faces. The $\theta_{i,o}$ are defined on faces for conservation
so the boundedness properties of this scheme require thought. This
can be re-arranged to give:
\begin{equation}
\psi_{C}^{n+1}=\gamma\psi_{C}^{n}+{\displaystyle \sum_{_{i\in\text{in}}}}\alpha_{i}\psi_{i}^{n}+\sum_{_{i\in\text{in}}}\beta_{i}\psi_{i}^{n+1}\label{eq:convexScheme}
\end{equation}
where
\begin{eqnarray*}
\alpha_{i} & = & \frac{\frac{\Delta t}{V_{C}}\left(1-\theta_{i}\right)U_{i}}{1+\frac{\Delta t}{V_{C}}{\displaystyle \sum_{o\in\text{out}}}\theta_{o}U_{o}}\text{ for each }i\\
\beta_{i} & = & \frac{\frac{\Delta t}{V_{j}}\theta_{i}U_{i}}{1+\frac{\Delta t}{V_{C}}{\displaystyle \sum_{o\in\text{out}}}\theta_{o}U_{o}}\text{ for each }i\\
\gamma & = & \frac{1-\frac{\Delta t}{V_{C}}{\displaystyle \sum_{o\in\text{out}}}\left(1-\theta_{o}\right)U_{o}}{1+\frac{\Delta t}{V_{C}}{\displaystyle \sum_{o\in\text{out}}}\theta_{o}U_{o}}.
\end{eqnarray*}
$\alpha_{i}$, $\beta_{i}$ and $\gamma$ are all positive as long
as the $\theta_{0}$ are chosen to give
\begin{equation}
\frac{\Delta t}{V_{C}}{\displaystyle \sum_{o\in\text{out}}}\left(1-\theta_{o}\right)U_{o}\le1
\end{equation}
which can be accomplished by setting
\begin{equation}
\theta_{f}\ge1-\frac{1}{\frac{\Delta t}{V_{C}}{\displaystyle \sum_{o\in\text{out}}}U_{o}}
\end{equation}
for the cells either side of face $f$. Hence all $\psi$ are positive
at the next time step. If, in addition the flow is discretely divergence
free then:
\begin{equation}
\sum_{i\in\text{in}}U_{i}=\sum_{o\in\text{out}}U_{o}
\end{equation}
which implies
\begin{equation}
\sum_{i\in\text{in}}\alpha_{i}+\sum_{i\in\text{in}}\beta_{i}+\gamma=1\label{eq:convexParams}
\end{equation}
so from (\ref{eq:convexScheme}) $\psi_{C}^{n+1}$ is a convex combination
of $\psi_{C}^{n}$, $\psi_{i}^{n}$ and $\psi_{i}^{n+1}$. This in
fact proves that the scheme is globally bounded. This can be shown
by contradiction. If we assume that $\psi_{C}^{n+1}$ is the global
maximum at time $n+1$ and it is greater than $\psi_{j}^{n}$ for
all cells $j$ in the mesh then (\ref{eq:convexScheme}) and (\ref{eq:convexParams})
cannot both hold for cell $C$. It is necessary for this scheme to
be bounded as it is used as the bounded scheme for the flux corrected
transport (section \eqref{subsec:FCT}).

\section{Stability Analysis of the Second-Order Implicit/Explicit Scheme\label{sec:appxStability}}

MPDATA is a non-linear scheme but the infinite gauge version becomes
the linear Lax-Wendroff scheme and so von-Neumman stability analysis
can be applied. In one dimension, for constant velocity, $u>0$, and
constant Courant number $c=u\Delta t/\Delta x$, the one-dimensional
implicit/explicit Lax-Wendroff scheme is
\begin{eqnarray}
\psi_{j}^{n+1} & = & \psi_{j}^{n}-c\left(1-\theta\right)\left(\psi_{j}^{n}-\psi_{j-1}^{n}\right)-c\theta\left(\psi_{j}^{n+1}-\psi_{j-1}^{n+1}\right)\label{eq:LWtheta}\\
 & - & \frac{c}{2}\left(1-\chi c\right)\left(\psi_{j+1}^{n}-2\psi_{j}^{n}+\psi_{j-1}^{n}\right)\nonumber 
\end{eqnarray}
where we showed in section \eqref{subsec:MPDATA_theta} that $\chi=1-2\theta$
gives second-order accuracy. From appendix \eqref{sec:appxBounded}
we can see that we need $\theta>1-1/c$ for stability of the first-order
upwind part (the first two terms of (\eqref{eq:LWtheta})). In order
to revert to the explicit version for $c\le1$ and to transition smoothly
to the implicit version we use
\begin{equation}
\theta=\begin{cases}
0 & c\le1\\
1-\frac{1}{c} & c>1
\end{cases}
\end{equation}
and analyse (\eqref{eq:LWtheta}) separately for these two cases.
Considering a Fourier mode with wavenumber $k$, the amplification
factor, $A$, of (\eqref{eq:LWtheta}) is
\begin{equation}
A=\frac{1-c\left(\chi c-\theta\right)\left(1-\cos k\Delta x\right)-ic\left(1-\theta\right)\sin k\Delta x}{1+c\theta\left(1-\cos k\Delta x\right)+ic\theta\sin k\Delta x}.
\end{equation}
For $\theta=0$ and $\chi=1$ we recover the usual Lax-Wendroff stability
constraints of $c\in\left[-1,1\right]$. For $c\ge1$ and $\theta=1-\frac{1}{c}$
it can be shown that stability requires $\chi\in\left[0,\frac{2c-1}{c^{2}}\right]$.
The stability range for $\chi$ is compared with the second-order
requirement for $\chi$ in figure \eqref{fig:MPDATA_chi}. For behaviour
as close as possible to second-order for the maximum range of Courant
numbers and for stability we use:
\begin{equation}
\chi=\max\left(1-2\theta,\ 0\right).
\end{equation}

\begin{figure}
\noindent \begin{centering}
\includegraphics[width=0.75\textwidth]{figures/LWstability}
\par\end{centering}
\caption{Comparison of the stability limits, the second-order requirement and
the value of $\chi$ used for the MPDATA correction.\label{fig:MPDATA_chi}}
\end{figure}

\end{document}
